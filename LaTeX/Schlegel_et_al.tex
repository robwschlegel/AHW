\documentclass[a4paper,10pt,review]{elsarticle}

\usepackage{lineno,hyperref} % Use this line to activate reference hyperlinks
% \usepackage{lineno} % Use this line to deactivate reference hyperlinks for ease of reviewing
\modulolinenumbers[5]

% START: Inserted by AJS
\frenchspacing
\usepackage{ifxetex}
\ifxetex
  \usepackage{fontspec}
  \defaultfontfeatures{Ligatures=TeX} % To support LaTeX quoting style
  \setromanfont{Hoefler Text}
  % \setmainfont[Ligatures=TeX]{Palatino}
\else
  \usepackage[T1]{fontenc}
  \usepackage[utf8]{inputenc}
  \usepackage{lmodern}
  \usepackage{textcomp} % directly use the degree (and some other) symbol
\fi

\usepackage{fixltx2e}
\usepackage[]{graphicx}
\usepackage{wrapfig}
\usepackage{lscape}
\usepackage{rotating}
\usepackage{epstopdf}
\usepackage{ragged2e}  % for '\RaggedRight' macro (allows hyphenation)
\usepackage[pdftex]{color}
\usepackage[margin=2.75cm]{geometry}
\usepackage{upquote}
\usepackage{textgreek}
\usepackage{microtype} % place after fonts; even better typesetting for improved readability
\usepackage{xfrac} % nice fractions
\usepackage{booktabs} % nice tables without vertical lines
\setlength\heavyrulewidth{0.1em}
\setlength\lightrulewidth{0.0625em}
\usepackage[color=yellow, textsize=tiny]{todonotes}
\usepackage[font={small}, labelfont=bf]{caption} % tweaking the captions
\usepackage{gensymb}
\usepackage{amsmath,amssymb}
\usepackage{cleveref} % clever cross referencing figures and tables; last package to include
% END: Inserted by AJS
\usepackage{natbib}

\journal{Progress in Oceanography}

%%%%%%%%%%%%%%%%%%%%%%%
%% Elsevier bibliography styles
%%%%%%%%%%%%%%%%%%%%%%%
%% To change the style, put a % in front of the second line of the current style and
%% remove the % from the second line of the style you would like to use.
%%%%%%%%%%%%%%%%%%%%%%%

%% Numbered
%\bibliographystyle{model1-num-names}

%% Numbered without titles
% \bibliographystyle{model1a-num-names}

%% Harvard
% \bibliographystyle{model2-names.bst}\biboptions{authoryear}

%% Vancouver numbered
%\usepackage{numcompress}\bibliographystyle{model3-num-names}

%% Vancouver name/year
% \usepackage{numcompress}\bibliographystyle{model4-names}\biboptions{authoryear}

%% APA style
% \bibliographystyle{model5-names}\biboptions{authoryear}

%% AMA style
%\usepackage{numcompress}\bibliographystyle{model6-num-names}

%% `Elsevier LaTeX' style
% \bibliographystyle{elsarticle-num}
\bibliographystyle{elsarticle-harv}\biboptions{authoryear}
% \bibliographystyle{elsarticle-num-names}
%%%%%%%%%%%%%%%%%%%%%%%

\begin{document}

\begin{frontmatter}

\title{The state of the union: air-sea interactions during coastal marine heatwaves}

%% or include affiliations in footnotes:
\author[firstaddress]{Robert W. Schlegel\corref{mycorrespondingauthor}}
\cortext[mycorrespondingauthor]{Corresponding author}
\ead{3503570@myuwc.ac.za}
\author[secondaddress,thirdaddress]{Eric C. J. Oliver}
\author[fourthaddress]{Sarah Kirkpatrick}
\author[fifthaddress]{Andries Kruger}
\author[firstaddress]{Albertus J. Smit}


% \author[mysecondaryaddress]{Global Customer Service\corref{mycorrespondingauthor}}

\address[firstaddress]{Department of Biodiversity and Conservation Biology, University of the Western Cape, Private Bag X17, Bellville 7535, South Africa}

\address[secondaddress]{ARC Centre of Excellence for Climate System Science, Australia}

\address[thirdaddress]{Institute for Marine and Antarctic Studies, University of Tasmania, Hobart, Australia}

\address[fourthaddress]{UWA Oceans Institute and School of Plant Biology, The University of Western Australia, Crawley, 6009 Western Australia, Australia}

\address[fifthaddress]{SAWS, South Africa}

\begin{abstract}
As the study of extreme climatic events increases, it becomes necessary to document the history of these events more thoroughly. In addition to documentation it behooves us to investigate potential mechanistic causal pathways that may allow us to better forecast the occurrence of these devastating events. To this end we have taken oceanic and atmospheric reanalysis data to examine the state of the air and sea surrounding coastal areas along the coast of South Africa at time in which extreme events have been documented. It was found that X, Y and Z occurred often in tandem with coastal MHWs. This may be taken as the first step of a more in depth exploratory analysis between what may be a causal link in the air sea interaction at these mid-latitude locations.
\end{abstract}

\begin{keyword}
extreme events \sep air-sea interaction \sep remotely-sensed SST \sep \emph{in situ} data \sep climate change \sep nearshore
\end{keyword}

\end{frontmatter}

\linenumbers

\section{Introduction}
The negative impact of anthropogenically forced warming as a result of changing climates on both marine and terrestrial ecosystems has become increasingly better documented over the last few decades. The primary topic of focus for changing climates often manifests itself as linear increases in mean temperatures in distinct regions. Whereas these long term changes are important and are already having documented impacts on a myriad of systems identified as critically important \citep{IPCC2014}, the major impacts on humans and ecosystems in the present are due to extreme events \citep{Easterling2000}. Often unpredictable, cyclones, floods, heatwaves and cold-spells may begin and end before any warning systems may be of use. It is for this reason, and others, that more focus in climate change research is now being applied to the study of these extreme events \citep{Jentsch2007}.

Due to the currently sparse occurrence of such extreme events in time and space, very few have impacted areas in which long term ecological data were being sampled \emph{a priori}. Two well documented exceptions to this trend are the 2003 heatwave in the Mediterannean and the 2011 heatwave off the west coast of Australia. The 2003 Mediterannean heatwave has been documented to have negatively impacted as much as 80\% of the Gorgonian fan colonies there \citep{Garrabou2009}, and the 2011 Western Australia heatwave is now known to have caused a permanent 100km range contraction of the ecosytstem forming kelp species \emph{Ecklonia radiata} in favour of the tropacalisation of reef fishes and seaweed turfs \citep{Wernberg2016}.

Both of these events are classified as 'marine heatwaves', which differ slightly from the traditional definition of heatwaves that was orginally developed for atmospheric events \citep{Perkins2013}. Here we make use of the definition for marine heatwaves given in \citet{Hobday2016} as ``a prolonged discrete anomalously warm water event that can be described by its duration, intensity, rate of evolution, and spatial extent.'' The characterisation of these events in this manner allows investigators from anywhere in the world to compare and classify events using common statistical properties.

The transport of warm water onto coastal areas is responsible for the large scale MHW tha happened off Western Australia in 2011 \citep{Feng2013, Benthuysen2014}. Recent research into the development of a mechanistic understanding between local- \emph{vs.} broad-scale influences on the formation of extreme events at coastal localities has revealed that meso-scale forcing of oceanic conditions onto the coast occurs far les than hypothesised \citep{Schlegel2016}. It is therefore necessary to consider additional mechanisms that may be responsible for these events. Air-sea interactions have been a focus of study for decades \citep{Frankignoul1985}, with mixed results. Whereas interactions are often detectable at high latitudes, mid latitude relationships between air and sea are much more tenuous \citep{Krishnamurti1988}. Equation 1 in \citet{Deser2010} shows the process through which the upper mixed layer in the open ocean is effected by atmospheric and oceanic process. Unfortunately this process does not appear to apply to the coastal regions of the world, of which little is yet understood of the mechanistic processes driving the extreme events observed there. In certain special instances, such as the 2003 heatwave over the Mediterannen described in \citet{Garrabou2009} a clear connection may be drawn between the air and sea. This is however an exception to the norm as most bodies of water are not subject to static atmospheric and oceanic conditions. One reason given for the lack of apparent air-sea interactions at mid-latitudes is that the coupling of these two media drives an increase in the variability of both, inhibiting heat flux from one to the other \citep{Barsugli1998}.

An earlier version of this manuscript sought to compare the co-occurrence of MHWs and AHWs measured \emph{in situ} along the coastline of South Africa via the same methodology outlined in \citet{Schlegel2016}. The rates of co-occurrence for extreme events between these media were found to be lower than those found for nearshore and offshore ocean water. It was therefore decided to create an index of image showing synoptically the mean air-sea state during the occurrence of coastal MHWs. The temperature dataset used for the calculation of the MHWs consisted of daily temperautre records collected \emph{in situ} at dozens of locations. The state of the sea, both SST and surface currents, was determined with BRAN. The state of the air, temperaure and wind, was determined with ERA-Interim. The aim of the application of these datasets to this investigations was to visually search for broadscale patterns in the air and/ or sea that occur at similar times to MHWs at coastal localities. We hypothised that i) similarities in the synoptic view of air and sea would reveal similarities between the largest MHWs detected in the dataset; ii) these similarities would be greater for the sea than the air; and iii) these observed similarities would aid in the development of a broader mechanistic understanding of the relationship between coastal MHWs and meso-scale forcing.

\section{Methods}
\subsection{Data}
Of the four datasets used for this investigation, the coastal seawater temperature data were acquired from the South African Coastal Temperature Network (SACTN, https://robert-schlegel.shinyapps.io/SACTN/). The SACTN data are contributed by seven different organizations and are collected \emph{in situ} with a mixture of hand-held alchol & mercury theromters as well as digital underwater temperature recorders (UTRs). This data set currently consists of XXX daily time series, with a mean duration of XXX years. This means that many of the time series in this dataset are shorter than the 30 year minimum proscribed for the characterization of MHWs \citep{Hobday2016}, with many having gaps of missing data, too. It is however deemed necessary to use these data when investigating extreme events along the coast as satellite derived sea surface temperature (SST) values along the coast have been shown to display large biases \citep{Smit2013} or capture minimum and maximum temperatures poorly \citep{Smale2009, Castillo2010}. All of the \emph{in situ} time series from the SACTN under ten years in length and/ or missing more than 10\% of their daily temperature measurement were excluded from use in this study. These subsetted time series numbered XXX in size, with a mean length of XXX. \Cref{tableS1} shows the metadata for the SACTN time series used in this study.

\section{Results}

\section{Discussion}

\section{Conclusion}

\section*{Acknowledgements}
We would like to thank DAFF, DEA, EKZNW, KZNSB, SAWS and SAEON for contributing all of the raw data used in this study. Without it, this article and the South African Coastal Temperature Network (SACTN) would not be possible. This research was supported by NRF Grant (CPRR14072378735) and by the Australian Research Council (FT110100174). This paper makes a contribution to the objectives of the Australian Research Council Centre of Excellence for Climate System Science (ARCCSS). The authors report no financial conflicts of interests. The data and analyses used in this paper may be found at https://github.com/schrob040/MHW. The Bluelink ocean data products were provided by CSIRO. Bluelink is a collaboration involving the Commonwealth Bureau of Meteorology, the Commonwealth Scientific and Industrial Research Organisation and the Royal Australian Navy.

\section*{References}


% Eric's paper outlining the methodology
% Oliver, E. C. J., V. Lago, N. J. Holbrook, S. D. Ling, C. N. Mundy, A. J. Hobday (2017), Eastern Tasmania Marine Heatwave Atlas, Institute for Marine and Antarctic Studies, University of Tasmania. doi: 10.4226/77/587e97d9b2bf9. http://metadata.imas.utas.edu.au/geonetwork/srv/eng/metadata.show?uuid=20188863-0af6-4032-98f8-def671cdaa58

% Citing ERA-interim
% http://onlinelibrary.wiley.com/doi/10.1002/qj.828/abstract

% Disable the following line when wanting to repopulate the .bbl file from the AHW.bib file
%\bibpunct{(}{)}{;}{a}{}{,} % Not certain this line is necessary...

\bibliography{AHW} % Comment out when manually copying the references from the .bbl file
% Delete all of the following when using the AHW.bib file with the above line
% No one here but us chickens...
% Delete the above line when using the AHW.bib file instead of copying in the .bbl file

\end{document}